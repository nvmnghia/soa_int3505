\documentclass{article}

\usepackage{hyperref}
\usepackage{indentfirst}

% References
\usepackage{biblatex}
\addbibresource{references.bib}


\author{Nguyễn Việt Minh Nghĩa \\ \href{mailto:nvmnghia@gmail.com}{nvmnghia@gmail.com}}

\date{11/03/2020}

\title{Kiến trúc hướng dịch vụ INT3505 \\ Bài tập lớn số một \\ Tìm hiểu các mô hình của điện toán đám mây}

\begin{document}

\maketitle

\section{Định nghĩa}

\subsection{Tiền đề}

"Cloud computing" là chủ đề nhận được rất nhiều sự quan tâm trong ngành công
nghệ thông tin. Từ khóa này thậm chí vượt ra khỏi chuyên ngành, trở thành một
buzzword thông dụng. Giống với nhiều thuật ngữ lớn khác như AI, Big data, điện
toán đám mây không có một định nghĩa cụ thể và thống nhất.

Với lập trình viên, điện toán đám mây đưoc hiểu đơn giản là việc \emph{cung cấp
tài nguyên tính toán và lưu trữ qua mạng}. Để hiểu phần nào định nghĩa này, ta
cần so sánh cách điện toán đám mây với cách sử dụng truyền thống tài nguyên máy
tính.

Trước đây, hai loại tài nguyên này thường đưọc đặt \emph{tại địa điểm} phát
triển hoặc sử dụng (on-premise). Việc đặt tài nguyên máy tính tại địa điểm xuất
phát từ nhu cầu thiết yếu về mặt quản trị, bảo mật (giữ quyền kiểm soát vật lý),
nhưng cũng có lí do về giới hạn công nghệ (tài nguyên đặt ở xa thì sử dụng chậm
hơn). Điều này dẫn tới việc công ty phải trả phí cho rất nhiều dịch vụ đi kèm,
chủ yếu gồm lắp đặt và bảo trì. Quan trọng hơn, người dùng trực tiếp (người lập
trình, người sử dụng, người vận hành) phần nào vẫn phải bận tâm đến những vấn đè
này, vốn không liên quan đến chuyên môn của họ, gây cản trở, tốn thời gian làm
việc không cần thiết. Nói ngắn gọn, bài toán thực tế đặt ra là tài nguyên cần
đưọc cung cấp mà người dùng chỉ cần sử dụng trực tiếp, không phải lo việc bảo
trì. Điện toán đám mây là một lời giải cho nhu cầu này.

Điện toán đám mây bắt nguồn từ ý tưởng rằng vị trí của dữ liệu và máy tính cần
\emph{trong suốt với với ngưòi dùng}, bằng cách đặt chúng trên máy chủ
\cite{SES2006}. Ngưòi dùng chỉ cần kết nối đến máy chủ đó để truy cập và sử dụng
tài nguyên, không cần quan tâm đến các công việc bảo trì như trước. Công ty cũng
không phải trả các khoản phí bảo trì, thậm chí không cần quan tâm đến nâng cấp.
Tất cả chi phí được gộp lại trả định kì theo dạng thuê bao, với hợp đồng rõ
ràng. Với ý tưởng này, tài nguyên tính toán và lưu trữ trở thành một loại dịch
vụ, giống với nước và điện.

Vấn đề với cách hiểu trên là sự thiếu chính xác, thể hiện ở các điểm sau:

\begin{itemize}
    \item Với các nhà khoa học máy tính, tính chính xác thể hiện tính hàn lâm;
    họ không cháp nhận một định nghĩa qua loa.
    \item Nó không phân biệt rõ ràng điện toán đám mây với một số dịch vụ đã
    từng tồn tại trước đó như máy chủ riêng áo (VPS).
    \item Nó khó tránh bị lạm dụng trong quảng cáo.
    \begin{itemize}
        \item Nhiều công ty, chủ yếu trong ngành dịch vụ máy chủ và ERP, trong
        đó có Dell, IBM, Oracle, dán mác "đám mây" cho các sản phẩm cũ của họ
        (cloudwashing) \cite{MITTR2011}.
        \item  Dell từng cố đăng kí bản quyền cụm từ "điện toán đám
        mây".
    \end{itemize}
\end{itemize}

Do vậy, cần thiết có một định nghĩa chính xác về điện toán đám mây. Phần tiếp
theo trình bày một định nghĩa như vậy của NIST, được công nhận và trích dẫn rộng
rãi, ra đời 5 năm sau khi điện toán đám mây trở nên phổ biến.

\printbibliography

\end{document}