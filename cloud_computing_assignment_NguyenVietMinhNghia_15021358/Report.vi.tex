\documentclass{article}

\usepackage{hyperref}
\usepackage{indentfirst}

% References
\usepackage{biblatex}
\addbibresource{references.bib}


\author{Nguyễn Việt Minh Nghĩa \\ \href{mailto:nvmnghia@gmail.com}{nvmnghia@gmail.com}}

\date{11/03/2020}

\title{Kiến trúc hướng dịch vụ INT3505 \\ Bài tập lớn số một \\ Tìm hiểu các mô hình của điện toán đám mây}

\begin{document}

\maketitle

\section{Định nghĩa}

\subsection{Tiền đề}

"Cloud computing" là chủ đề nhận được rất nhiều sự quan tâm trong ngành công
nghệ thông tin. Từ khóa này thậm chí vượt ra khỏi chuyên ngành, trở thành một
buzzword thông dụng. Giống với nhiều thuật ngữ lớn khác như AI, Big data, điện
toán đám mây không có một định nghĩa cụ thể và thống nhất.

Với lập trình viên, điện toán đám mây đưoc hiểu đơn giản là việc \emph{cung cấp
tài nguyên tính toán và lưu trữ qua mạng}. Để hiểu phần nào định nghĩa này, ta
cần so sánh cách điện toán đám mây với cách sử dụng truyền thống tài nguyên máy
tính.

Trước đây, hai loại tài nguyên này thường đưọc đặt \emph{tại địa điểm} phát
triển hoặc sử dụng (on-premise). Việc đặt tài nguyên máy tính tại địa điểm xuất
phát từ nhu cầu thiết yếu về mặt quản trị, bảo mật (giữ quyền kiểm soát vật lý),
nhưng cũng có lí do về giới hạn công nghệ (tài nguyên đặt ở xa thì sử dụng chậm
hơn). Điều này dẫn tới việc đến hai hậu quả:

\begin{itemize}
    \item Công ty phải trả phí cho rất nhiều dịch vụ đi kèm, chủ yếu gồm lắp đặt
    và bảo trì.
    \item Người dùng trực tiếp (người lập trình, người vận hành) phần nào vẫn
    phải bận tâm đến những vấn đè ngoài chuyên môn (bảo trì, sử dụng).
\end{itemize}

Cả công ty và người sử dụng đều mất tiền và thời gian, công sức không cần thiết.
Nói ngắn gọn, bài toán thực tế đặt ra là tài nguyên cần đưọc cung cấp mà người
dùng chỉ cần sử dụng trực tiếp, không phải lo việc bảo trì. Điện toán đám mây là
một lời giải cho nhu cầu này.

Điện toán đám mây bắt nguồn từ ý tưởng rằng vị trí và các hoạt động bảo trì của
tài nguyên cần \emph{trong suốt với với ngưòi dùng}, bằng cách đặt chúng trên
máy chủ \cite{SES2006}. Ý tưởng này giải quyết được cả hai vấn đề đề cập ở trên:

\begin{itemize}
    \item Ngưòi dùng chỉ cần kết nối đến máy chủ đó để truy cập và sử dụng tài
    nguyên, không cần quan tâm đến các công việc bảo trì như trước.
    \item Công ty chỉ cần trả chi phí định kì; phí này bao gồm tất cả chi phí
    mua sắm, vận hành, nâng cấp.
\end{itemize}

Với ý tưởng này, tài nguyên tính toán và lưu trữ trở thành một loại dịch vụ thuê
bao, giống với nước, điện và sóng điện thoại. Tổng chi phí sở hữu (TCO) của
doanh nghiệp được cắt giảm hoàn toàn.

Vấn đề với cách hiểu trên là sự thiếu chính xác, thể hiện ở một số điểm sau:

\begin{itemize}
    \item Thiếu tính hàn lâm
    \item Không phân biệt rõ ràng điện toán đám mây với một số dịch vụ đã
    từng tồn tại trước đó như máy chủ riêng áo (VPS)
    \item Khó tránh bị lạm dụng trong quảng cáo
    \begin{itemize}
        \item Nhiều công ty, chủ yếu trong ngành dịch vụ máy chủ và ERP, trong
        đó có Dell, IBM, Oracle, dán mác "đám mây" cho các sản phẩm cũ của họ
        (cloudwashing) \cite{MITTR2011}.
        \item  Dell từng cố đăng kí bản quyền cụm từ "điện toán đám
        mây".
    \end{itemize}
\end{itemize}

Do vậy, cần thiết có một định nghĩa chính xác về điện toán đám mây. Phần tiếp
theo trình bày một định nghĩa như vậy của NIST, được công nhận và trích dẫn rộng
rãi, ra đời 5 năm sau khi điện toán đám mây trở nên phổ biến.

\subsection{Định nghĩa của NIST}

NIST định nghĩa điện toán đám mây như sau \cite{NIST2011}:

Điện toán đám mây là mô hình cho phép truy cập một cách thuận tiện, theo yêu cầu
vào một bể tài nguyên dùng chung. Tài nguyên đó có thể là mạng, máy chủ, thiết
bị lưu trữ, ứng dụng, dịch vụ. Tài nguyên này có thể được cấu hình, có thể được
cung cấp và hoàn trả nhanh chóng mà không cần can thiệp từ nhà cung cấp dịch vụ.
Mô hình này gồm năm đặc tính:

\begin{itemize}
    \item Tự phục vụ theo nhu cầu: Người dùng có thể sử dụng tài nguyên theo nhu
    cầu mà không cần liên hệ trước với nhà cung cấp dịch vụ.
    \item Truy cập qua mạng: Tài nguyên có thể đưọc sử dụng qua mạng, bằng các
    thiết bị đầu cuối thông thường, từ điện thoại đến máy tính.
    \item Tài nguyên dùng chung: Tài nguyên của nhà cung cấp đưọc gộp lại và cho
    mọi người dùng chung, sao cho sự riêng tư vẫn được đảm bảo. Dịch vụ cần đạt
    đưọc sự trong suốt về vị trí, theo nghĩa là người dùng không cần kiểm soát
    hay biết vị trí địa lí của tài nguyên, nhưng có thể chỉ định vị trí ở mức
    cao hơn (quốc gia, địa điểm, trung tâm dữ liệu,...).
    \item Cung cấp tài nguyên linh hoạt: tài nguyên có thể được cung cấp cho
    người dùng một cách tự động, không cần người dùng biết, tùy theo mức tải.
    Với người dùng, dịch vụ có vẻ vô tận.
    \item Dịch vụ được điều tiết: Hệ thống tự động điều khiển và tối ưu việc sử
    dụng tài nguyên bằng các công cụ đo lường. Chi phí sử dụng được tính theo
    cách thông số đo đạc này. Thông tin sử dụng có thể đưọc theo dõi, báo cáo
    cho cả người dùng và nhà cung cấp dịch vụ.
\end{itemize}

\section{}

\printbibliography

\end{document}